% Copyright 2017 Beckman Coulter, Inc.
%
% Permission is hereby granted, free of charge, to any person
% obtaining a copy of this software and associated documentation files
% (the "Software"), to deal in the Software without restriction,
% including without limitation the rights to use, copy, modify, merge,
% publish, distribute, sublicense, and/or sell copies of the Software,
% and to permit persons to whom the Software is furnished to do so,
% subject to the following conditions:
%
% The above copyright notice and this permission notice shall be
% included in all copies or substantial portions of the Software.
%
% THE SOFTWARE IS PROVIDED "AS IS", WITHOUT WARRANTY OF ANY KIND,
% EXPRESS OR IMPLIED, INCLUDING BUT NOT LIMITED TO THE WARRANTIES OF
% MERCHANTABILITY, FITNESS FOR A PARTICULAR PURPOSE AND
% NONINFRINGEMENT. IN NO EVENT SHALL THE AUTHORS OR COPYRIGHT HOLDERS
% BE LIABLE FOR ANY CLAIM, DAMAGES OR OTHER LIABILITY, WHETHER IN AN
% ACTION OF CONTRACT, TORT OR OTHERWISE, ARISING FROM, OUT OF OR IN
% CONNECTION WITH THE SOFTWARE OR THE USE OR OTHER DEALINGS IN THE
% SOFTWARE.

\chapter {Application}\label{chap:application}

\section {Introduction}

The application\index{application} is a single gen-server named
\code{application} that manages the lifetime of the program.  It
links to a process, typically the root supervisor, and shuts down the
program when requested by \code{application:shutdown} or when the
linked process dies.

\section {Theory of Operation}

\paragraph* {state}\index{application!state}
The application state is the process returned by the \var{starter} of
\code{application:start}. It is typically the root supervisor. We
refer to this variable as \var{process}. It may also be \code{\#f}
after \code{handle-info} receives the exit message for the process.

\genserver{application}{init} The application \code{init} procedure
takes a \var{starter} procedure. It calls \code{(\var{starter})} and
checks the return value $r$. If $r$ = \code{\#(ok \var{process})},
it links to \var{process}, traps exits so that it receives exit
messages from \var{process} and \code{application:shutdown}, and
returns \code{\#(ok \var{process})}. If $r$ = \code{\#(error
  \var{reason})}, it returns \code{\#(stop \var{reason})}.

\genserver{application}{terminate} The application \code{terminate}
procedure shuts down \var{process}. When \var{process} is not
\code{\#f}, it kills \var{process} with reason \code{shutdown} and
waits indefinitely for it to terminate. Then it calls
\code{(exit-process \var{exit-code})}, where \var{exit-code} is
initially 2 but set to the value passed to
\code{application:shutdown}. In this way, the exit code can be used
to determine if the application shut down normally.

\genserver{application}{handle-call} The application
\code{handle-call} procedure raises an exception on all messages.

\genserver{application}{handle-cast} The application
\code{handle-cast} procedure raises an exception on all messages.

\genserver{application}{handle-info} The application
\code{handle-info} procedure handles the following message:\antipar

\begin{itemize}
\item \code{\#(EXIT \var{p} \var{reason})}: If \var{p} =
  \var{process}, return \code{\#(stop \var{reason} \#f)}. Otherwise,
  return \code{\#(stop \var{reason} \var{process})}.
\end{itemize}

\index{application!exit-process@\code{exit-process}}
\begin{procedure}
  \code{(exit-process \var{exit-code})}
\end{procedure}
\returns{} never

The \code{exit-process} procedure flushes the console output port,
ignoring any exceptions, and then calls \code{(ExitProcess
  \var{exit-code})}.

\section {Programming Interface}

\defineentry{application:start}
\begin{procedure}
  \code{(application:start \var{starter})}
\end{procedure}
\returns{} \code{ok}

The \code{application:start} procedure calls
\code{(gen-server:start 'application \var{starter})}. If it returns
\code{\#(ok \_)}, \code{application:start} returns \code{ok}. If
it returns \code{\#(error \var{reason})}, \code{application:start}
calls \code{(console-event-handler \#(application-start-failed
  \var{reason}))} and \code{(exit-process 1)}.

\defineentry{application:shutdown}
\begin{procedure}
  \code{(application:shutdown \opt{\var{exit-code}})}
\end{procedure}
\returns{} unspecified

The \code{application:shutdown} procedure kills the
\code{application} process with reason \code{shutdown}. The
\var{exit-code} defaults to 0, indicating normal shutdown. The
procedure does not wait for the \code{application} process to
terminate so that it can be called from a process managed by the
supervision hierarchy without causing a deadlock on shutdown. If the
\code{application} process does not exist,
\code{application:shutdown} calls \code{(exit-process
  \var{exit-code})}.
